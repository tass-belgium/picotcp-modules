It is assumed that all examples include the appropriate header files
and a \textbf{main} routine that calls the \texttt{app$\_$x} functions to initialize
the example.

The most common header files are:
\begin{verbatim}
#include "pico_http_client.h"
#include "pico_http_server.h"
#include "pico_http_util.h"
\end{verbatim}

\section{HTTP Client example}
\begin{verbatim}
static char *uri_filename = NULL;

static int8_t http_save_file(void *data, int len)
{
    int32_t fd = open(uri_filename, O_WRONLY |O_CREAT | O_TRUNC, 0660);
    int32_t w, e;
    if (fd < 0)
        return fd;
    printf("Saving data to : %s\n",uri_filename);
    w = write(fd, data, len);
    e = errno;
    close(fd);
    errno = e;
    return w;
}

void wget_callback(uint16_t ev, uint16_t conn)
{
    char data[1024 * 1024]; // MAX: 1M
    static int _length = 0;
    int32_t ret = 0;

    if(ev & EV_HTTP_CON)
    {
        printf(">>> Connected to the client \n");
        /* you can let the client use the default generated header
        or you can create you own string header and send it
        via pico_http_client_send_raw(...)*/
        pico_http_client_send_get(conn, NULL, HTTP_CONN_CLOSE);
    }

    if (ev & EV_HTTP_WRITE_SUCCESS)
    {
        printf("The request write was successful ... \n");
    }

    if (ev & EV_HTTP_WRITE_PROGRESS_MADE)
    {
        uint32_t total_bytes_written = 0;
        uint32_t total_bytes_to_write = 0;
        printf("The request write has made progress ... \n");
        ret = pico_http_client_get_write_progress(conn, &total_bytes_written,
                                                  &total_bytes_to_write);
        if (ret < 0)
        {
            printf("Nothing to write anymore\n");
        }
        else
        {
            printf("Total_bytes_written: %d, Total_bytes_to_write: %d\n",
                   total_bytes_written, total_bytes_to_write);
        }
    }

    if (ev & EV_HTTP_WRITE_FAILED)
    {
        printf("The request write has failed ... \n");
    }

    if(ev & EV_HTTP_REQ)
        {
        struct pico_http_header * header = pico_http_client_read_header(conn);
        printf("Received header from server...\n");
        printf("Server response : %d\n",header->response_code);
        printf("Location : %s\n",header->location);
        printf("Transfer-Encoding : %d\n",header->transfer_coding);
        printf("Size/Chunk : %d\n",header->content_length_or_chunk);
    }

    if(ev & EV_HTTP_BODY)
    {
        uint32_t len;
        printf("Reading data...\n");
        uint8_t body_read_done = 0;
        /*
        Data is passed to you without you worrying if the transfer is
        chunked or the content-length was specified.
        */
        while((len = pico_http_client_read_body(conn,data + _length,1024,
                                                &body_read_done)))
        {
        _length += len;
        }
    }

    if(ev & EV_HTTP_CLOSE)
    {
        struct pico_http_header * header = pico_http_client_read_header(conn);
        int32_t len;
        uint8_t body_read_done = 0;
        printf("Connection was closed...\n");
        printf("Reading remaining data, if any ...\n");

        while((len = pico_http_client_read_body(conn,data,1000u,
                                                &body_read_done)) && len > 0)
        {
            _length += len;
        }
        printf("Read a total data of : %d bytes \n",_length);

        if(header->transfer_coding == HTTP_TRANSFER_CHUNKED)
        {
            if(header->content_length_or_chunk)
            {
                printf("Last chunk data not fully read !\n");
                exit(1);
            }
            else
            {
                printf("Transfer ended with a zero chunk! OK !\n");
            }
        }
        else
        {
            if(header->content_length_or_chunk == _length)
            {
                printf("Received the full : %d \n",_length);
            }
            else
            {
                printf("Received %d , waiting for %d\n",_length,
                       header->content_length_or_chunk);
                exit(1);
            }
        }
        if (!uri_filename)
        {
            printf("Failed to get local filename\n");
            exit(1);
        }
        if (http_save_file(data, _length) < _length)
        {
            printf("Failed to save file: %s\n", strerror(errno));
            exit(1);
        }
        pico_http_client_close(conn);
        exit(0);
    }

    if(ev & EV_HTTP_ERROR)
    {
        printf("Connection error (probably dns failed : check the routing table),
               trying to close the connection";
        pico_http_client_close(conn);
        exit(1u);
    }
    if(ev & EV_HTTP_DNS)
    {
        printf("The DNS query was successful ... \n");
    }
}

void app_wget(char *arg[])
{
    char * uri;
    cpy_arg(&uri, arg[0]);
    if(!uri)
    {
        fprintf(stderr, " wget expects the uri
                        to be received\n");
        exit(1);
    }
    // when opening the http client it will
    // internally parse the uri passed
    if(pico_http_client_open(uri, wget_callback) < 0)
    {
        fprintf(stderr," error opening the uri : %s,
                please check the format\n",uri);
        exit(1);
    }
    uri_filename = basename(uri);
}
\end{verbatim}


\section{HTTP Server example}
\begin{verbatim}
#define SIZE 4*1024

void serverWakeup(uint16_t ev,uint16_t conn)
{
  static FILE * f;
  char buffer[SIZE];

  if(ev & EV_HTTP_CON)
  {
      printf("New connection received....\n");
      pico_http_server_accept();
  }

  if(ev & EV_HTTP_REQ) // new header received
  {
    int read;
    char * resource;
    printf("Header request was received...\n");
    printf("> Resource : %s\n",pico_http_get_resource(conn));
    resource = pico_http_get_resource(conn);

    if(strcmp(resource,"/")==0 || strcmp(resource,"index.html")
         == 0 || strcmp(resource,"/index.html") == 0)
    {
          // Accepting request
          printf("Accepted connection...\n");
          pico_http_respond(conn,HTTP_RESOURCE_FOUND);
          f = fopen("test/examples/index.html","r");

          if(!f)
          {
            fprintf(stderr,"Unable to open the file
                        /test/examples/index.html\n");
            exit(1);
          }

          read = fread(buffer,1,SIZE,f);
          pico_http_submit_data(conn,buffer,read);
    }
    else
    { // reject
      printf("Rejected connection...\n");
      pico_http_respond(conn,HTTP_RESOURCE_NOT_FOUND);
    }

  }

  if(ev & EV_HTTP_PROGRESS) // submitted data was sent
  {
    uint16_t sent, total;
    pico_http_get_progress(conn,&sent,&total);
    printf("Chunk statistics : %d/%d sent\n",sent,total);
  }

  if(ev & EV_HTTP_SENT) // submitted data was fully sent
  {
    int read;
    read = fread(buffer,1,SIZE,f);
    printf("Chunk was sent...\n");
    if(read > 0)
    {
        printf("Sending another chunk...\n");
        pico_http_submit_data(conn,buffer,read);
    }
    else
    {
        printf("Last chunk !\n");
        pico_http_submit_data(conn,NULL,0);// send the final chunk
        fclose(f);
    }
  }

  if(ev & EV_HTTP_CLOSE)
  {
    printf("Close request...\n");
    pico_http_close(conn);
  }

  if(ev & EV_HTTP_ERROR)
  {
    printf("Error on server...\n");
    pico_http_close(conn);
  }
}
/* simple server example that serves the index(.html) page */
void app_httpd(char *arg)
{
  /* transfer encoding with this server is always chunked and you can
     submit chunks to the client, without needing to
     specify the content-length of the
     body response */
  if( pico_http_server_start(0,serverWakeup) < 0)
  {
    fprintf(stderr,"Unable to start the server on port 80\n");
  }
}
\end{verbatim}
